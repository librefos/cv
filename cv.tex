%--- The MIT License (MIT) ---------------------------------------------
%
% Copyright (c) 2019 Jan Küster
% Copyright (c) 2026 Thiago C Silva
%
% Permission is hereby granted, free of charge, to any person obtaining
% a copy of this software and associated documentation files (the
% "Software"), to deal in the Software without restriction, including
% without limitation the rights to use, copy, modify, merge, publish,
% distribute, sublicense, and/or sell copies of the Software, and to
% permit persons to whom the Software is furnished to do so, subject to
% the following conditions:
%
% The above copyright notice and this permission notice shall be
% included in all copies or substantial portions of the Software.
%
% THE SOFTWARE IS PROVIDED "AS IS", WITHOUT WARRANTY OF ANY KIND,
% EXPRESS OR IMPLIED, INCLUDING BUT NOT LIMITED TO THE WARRANTIES OF
% MERCHANTABILITY, FITNESS FOR A PARTICULAR PURPOSE AND NONINFRINGEMENT.
% IN NO EVENT SHALL THE AUTHORS OR COPYRIGHT HOLDERS BE LIABLE FOR ANY
% CLAIM, DAMAGES OR OTHER LIABILITY, WHETHER IN AN ACTION OF CONTRACT,
% TORT OR OTHERWISE, ARISING FROM, OUT OF OR IN CONNECTION WITH THE
% SOFTWARE OR THE USE OR OTHER DEALINGS IN THE SOFTWARE.
%
%--------------------------------------------- The MIT License (MIT) ---

\documentclass[10pt, a4paper]{article}

%--- The MIT License (MIT) ---------------------------------------------
%
% Copyright (c) 2019 Jan Küster
% Copyright (c) 2026 Thiago C Silva
%
% Permission is hereby granted, free of charge, to any person obtaining
% a copy of this software and associated documentation files (the
% "Software"), to deal in the Software without restriction, including
% without limitation the rights to use, copy, modify, merge, publish,
% distribute, sublicense, and/or sell copies of the Software, and to
% permit persons to whom the Software is furnished to do so, subject to
% the following conditions:
%
% The above copyright notice and this permission notice shall be
% included in all copies or substantial portions of the Software.
%
% THE SOFTWARE IS PROVIDED "AS IS", WITHOUT WARRANTY OF ANY KIND,
% EXPRESS OR IMPLIED, INCLUDING BUT NOT LIMITED TO THE WARRANTIES OF
% MERCHANTABILITY, FITNESS FOR A PARTICULAR PURPOSE AND NONINFRINGEMENT.
% IN NO EVENT SHALL THE AUTHORS OR COPYRIGHT HOLDERS BE LIABLE FOR ANY
% CLAIM, DAMAGES OR OTHER LIABILITY, WHETHER IN AN ACTION OF CONTRACT,
% TORT OR OTHERWISE, ARISING FROM, OUT OF OR IN CONNECTION WITH THE
% SOFTWARE OR THE USE OR OTHER DEALINGS IN THE SOFTWARE.
%
%--------------------------------------------- The MIT License (MIT) ---
%--- Packages & Configuration ------------------------------------------

% Paper size and page margins
\usepackage[a4paper]{geometry}

% Customization of headers and footers
\usepackage{fancyhdr}

% Parallel column layouts
\usepackage{paracol}

% Tools for table column formatting
\usepackage{array}

% Improves text justification in narrow columns
\usepackage{microtype}

% Correct hyphenation of accented words
\usepackage[T1]{fontenc}

% Correct hyphenation for Portuguese
\usepackage[portuguese]{babel}

% Document main font to 'Raleway'
\usepackage[default]{raleway}

% Icon sets
\usepackage{fontawesome5}

% Extra font sizes
\usepackage{moresize}

% Provides \ifstrempty
\usepackage{etoolbox}

% Arithmetic in length commands
\usepackage{calc}

% Inclusion of external images
\usepackage{graphicx}

% Color management support
\usepackage{xcolor}

% Drawing the skill bars
\usepackage{tikz}

% Transparency in elements
\usepackage{transparent}

% Handles hyperlinks; 'hidelinks' removes the colored box
\usepackage[hidelinks]{hyperref}

% Generates QR codes from text/URLs
\usepackage{qrcode}

% Load specific TikZ libraries
\usetikzlibrary{shapes, backgrounds, mindmap, trees}

%------------------------------------------ Packages & Configuration ---
%--- Layout & Column ---------------------------------------------------

% Restore original margins
\geometry{top=1cm, bottom=6cm, left=1cm, right=1cm}

% 31% left column, 69% right column
\columnratio{0.31}

% Define space between columns
\setlength{\columnsep}{2.2em}

% Define the vertical rule between columns
\setlength{\columnseprule}{4pt}

%--------------------------------------------------- Layout & Column ---
%--- Colors & Fonts ----------------------------------------------------

% Define custom colors used throughout the document
\definecolor{maincol}{RGB}{25, 39, 209}
\definecolor{darkcol}{RGB}{43, 49, 108}
\definecolor{lightcol}{RGB}{245, 245, 245}

% Set the color of the column separator rule
\colseprulecolor{lightcol}

% Set the default font family to sans-serif
\renewcommand*\familydefault{\sfdefault}

% Remove the header and footer decorative lines
\renewcommand{\headrulewidth}{0pt}
\renewcommand{\footrulewidth}{0pt}

%---------------------------------------------------- Colors & Fonts ---
%--- New Commands ------------------------------------------------------

% Usage: \linkedin{username}
\newcommand*{\linkedin}[1]{https://linkedin.com/in/#1}

% Usage: \github{username}
\newcommand*{\github}[1]{https://github.com/#1}

% Usage: \vcenteredhbox{box}
\newcommand*{\vcenteredhbox}[1]{%
  \begingroup
    \sbox{0}{#1}%
    \parbox{\wd0}{\usebox{0}}%
  \endgroup
}

% Usage: \iconlayout{icon_name}{size}{color}{content_to_render}
\newcommand*{\iconlayout}[4]{%
  \vcenteredhbox{\icon{#1}{#2}{#3}}%
  \hspace{2pt}%
  \parbox{0.9\mpwidth}{#4}
}

% Usage: \icon{icon_name}{size}{color}
\newcommand*{\icon}[3]{%
  \makebox[#2pt][c]{\textcolor{#3}{\faIcon{#1}}}%
}

% Usage: \icontext{icon}{size}{text}{color}
\newcommand*{\icontext}[4]{%
  \iconlayout{#1}{#2}{#4}{%
    \textcolor{#4}{#3}%
  }%
}

% Usage: \iconhref{icon}{size}{text}{url}{color}
\newcommand*{\iconhref}[5]{%
  \iconlayout{#1}{#2}{#5}{%
    \href{#4}{\textcolor{#5}{#3}}%
  }%
}

% Usage: \iconemail{icon_name}{size}{email_address}{color}
\newcommand*{\iconemail}[4]{%
  \iconlayout{#1}{#2}{#4}{%
    \href{mailto:#3}{\textcolor{#4}{#3}}%
  }%
}

% Define the width for boxes: Line width minus the frame padding (x2)
\newcommand{\mpwidth}{\linewidth-2\fboxsep}

% Usage: \cvtext{content}
\newcommand{\cvtext}[1]{%
  \parbox{\mpwidth}{%
    #1%
  }%
}

% Usage: \cvlist{ \item ... }
\newcommand{\cvlist}[1]{%
  \begin{itemize}%
    #1%
  \end{itemize}%
}

% Usage: \cvsection{Title}
\newcommand{\cvsection}[1]{%
  \vspace{14pt}%
  \cvtext{%
    \textbf{\LARGE{\textcolor{darkcol}{\uppercase{#1}}}}\\[-4pt]%
    \textcolor{darkcol}{\rule{0.1\textwidth}{2pt}}%
  }%
  \vspace{14pt}%
}

% Usage: \cvmetaevent{Date}{Title}{Institution}{Description}
\newcommand{\cvmetaevent}[4]{%
  \cvtext{\textbf{#1}}\\[4pt]%
  \ifstrempty{#2}{}{%
    \cvtext{\textbf{\textcolor{darkcol}{#2}}}\\[4pt]%
  }%
  \ifstrempty{#3}{}{%
    \cvtext{\textcolor{darkcol}{#3}}\\[6pt]%
  }%
  \cvtext{#4}\\[8pt]%
}

% Usage: \cvskill{Skill Name}{Experience Level}{Percentage}
\newcommand{\cvskill}[3]{%
  \begin{tabular*}{\mpwidth}{@{}p{0.75\mpwidth}@{\extracolsep{\fill}}r@{}}
    \textcolor{black}{\textbf{#1}} & \textcolor{darkcol}{#2}%
  \end{tabular*}%
  \\[2pt]%
  \begin{tikzpicture}[scale=1,rounded corners=2pt,very thin]
    \fill [lightcol] (0,0) rectangle (\linewidth, 0.15);
    \fill [darkcol] (0,0) rectangle (#3\linewidth, 0.15);
  \end{tikzpicture}%
  \vspace{4pt}%
}

% Usage: \cvevent{Date}{Job Title}{Company}{Description}{Skills_List}
\newcommand{\cvevent}[5]{%
  \begin{minipage}{\mpwidth}
    \begin{tabular*}{\mpwidth}{@{}p{0.7\mpwidth}@{\extracolsep{\fill}}r@{}}
      \textcolor{darkcol}{\textbf{#2}} &
      \colorbox{black}{%
        \makebox[0.25\mpwidth][c]{\textcolor{white}{\textbf{#1}}}%
      }%
    \end{tabular*}%
    \\[2pt]%
    \textcolor{black}{\textbf{#3}}\\[6pt]%
    \ifstrempty{#4}{}{%
      \cvtext{#4}\\
    }%
  \end{minipage}%

  \ifstrempty{#5}{}{%
    \vspace{6pt}%
    \cvtext{\textbf{Principais Competências:}}%
    {#5}%
  }%
  \vspace{12pt}%
}

%------------------------------------------------------ New Commands ---


%--- Header & Footer ---------------------------------------------------

\pagestyle{fancy}

% Clear all default headers/footers
\fancyhf{}

% Left-aligned footer with contact info
\fancyfoot[L]{%
  \iconhref{github}{12}{@librefos}{\github{librefos}}{black} \\[6pt]
  \iconemail{envelope}{12}{librefos@hotmail.com}{black} \\[6pt]
  \icontext{map-marker-alt}{12}{Curvelo/MG, Brasil}{black} \\[12pt]
  \qrcode{\linkedin{librefos}}
}

\begin{document}

% Indentation at the beginning of each paragraph
\setlength{\parindent}{0mm}

% Start the two-column layout
\begin{paracol}{2}

%--------------------------------------------------- Header & Footer ---
%--- Left Column -------------------------------------------------------

\begin{leftcolumn}

\includegraphics[width=\linewidth]{assets/profile.jpeg}

\cvsection{Competências}

\cvskill{PHP}             {Pleno}  {1.00}
\cvskill{C}               {Júnior} {0.75}
\cvskill{Java}            {Júnior} {0.75}
\cvskill{Python}          {Júnior} {0.75}
\cvskill{SQL}             {Júnior} {0.75} \\

\cvskill{GNU/Linux}       {Pleno}  {1.00}
\cvskill{Shell Scripting} {Pleno}  {1.00}
\cvskill{Git}             {Pleno}  {1.00}
\cvskill{Docker/Podman}   {Júnior} {0.75} \\

\cvskill{System Admin}    {Pleno}  {1.00}
\cvskill{Web Performance} {Pleno}  {1.00}
\cvskill{Automation}      {Júnior} {0.75}
\cvskill{Cibersegurança}  {Júnior} {0.75}

\cvsection{Formação}

\cvmetaevent
  {2023 - 2027}
  {Bacharelado em Engenharia de Software}
  {UniCesumar}
  {Graduação focada na fundamentação teórica e prática da engenharia de
  software. O currículo abrange a Engenharia de Requisitos, Arquitetura
  de Software e Modelagem de Sistemas, integrando conceitos de
  Cibersegurança e Computação em Nuvem. A formação enfatiza o
  desenvolvimento de algoritmos, estruturas de dados e Programação
  Orientada a Objetos, preparando para a gestão de projetos de software
  e design de soluções escaláveis.}

\cvmetaevent
  {2023 - 2024}
  {Técnico em Desenvolvimento de Sistemas}
  {Senac Minas}
  {Habilitação técnica focada na implementação prática de sistemas
  computacionais. O curso abrange o ciclo completo de desenvolvimento
  para aplicações. A grade curricular prioriza a lógica de programação e
  codificação, juntamente com a modelagem e administração de Bancos de
  Dados Relacionais. Enfatiza a aplicação de metodologias ágeis, testes
  de software e versionamento de código em projetos integradores.}

\end{leftcolumn}

%------------------------------------------------------- Left Column ---
%--- Right Column ------------------------------------------------------

\begin{rightcolumn}

% Usage: \fcolorbox{frame_color}{bg_color}{content}
\fcolorbox{white}{black}{%
  \begin{minipage}[c][3.5cm][c]{\mpwidth}
    \begin{center}
      \HUGE{\textbf{\textcolor{white}{%
        \MakeUppercase{Thiago C. Silva}%
      }}}\\[-24pt]
      \textcolor{white}{\rule{0.1\textwidth}{1.25pt}}\\[4pt]
      \large{\textcolor{white}{Engenheiro de Software}}
    \end{center}
  \end{minipage}%
}\\[14pt]
\vspace{-12pt}

\cvsection{Perfil Profissional}

\cvtext{%
  Técnico em Desenvolvimento de Sistemas e graduando em Engenharia
  de Software com perfil híbrido entre Desenvolvimento e Administração de
  Sistemas. Adoto uma abordagem de ``Fundamentos'', unindo o rigor da
  engenharia tradicional à automação moderna para criar softwares seguros
  e auditáveis.\\[6pt]
  %
  Engajado com a filosofia de Software Livre, adoto a transparência e a
  colaboração como métodos de trabalho. Encaro a documentação técnica e
  a análise profunda de código-fonte como ferramentas indispensáveis para
  a garantia de qualidade.%
}

\cvsection{Experiência Profissional}

\cvevent
  {2023 - Jul 2025}
  {Engenheiro de Software \& Infraestrutura Web}
  {Flow Digital}
  {Responsável por garantir a disponibilidade e segurança dos serviços,
  traduzindo requisitos estratégicos de negócio em soluções robustas e
  de alta performance. Gestão do ciclo de vida de aplicações web, com
  foco em maximizar métricas de performance e segurança. Atuação na
  administração de servidores, eliminando gargalos, melhorando a
  eficiência e desempenho para atender demandas de marketing e SEO
  técnico.}
  {\cvlist{
    \item Otimização de performance (Core Web Vitals)
    \item Desenvolvimento e manutenção de aplicações web
    \item Linux \& Shell Scripting para automação
    \item PHP \& CMS Management
    \item Administração de servidores web
    \item Analytics \& SEO Técnico
    \item Melhoria mensurável em métricas de performance e SEO
  }}

\vspace{10pt}

\cvevent
  {Jul 2025 - Atual}
  {Desenvolvedor de Software \& Consultor Técnico}
  {Autônomo}
  {Desenvolvimento de soluções de software completas, com ênfase na
  qualidade de código, segurança e automação. Atuação consultiva para
  resolver problemas complexos de integração e infraestrutura para
  diversos clientes.}
  {\cvlist{
    \item Automação de processos e scripts de infraestrutura
    \item Docker \& Ferramentas de Containerização
    \item SQL \& Modelagem de Dados Relacional
    \item Git \& Metodologias Ágeis de entrega
    \item Soluções focadas em resolver a causa raiz dos problemas de negócio
  }}

\cvsection{Certificações}

\cvevent
  {Dezembro 2024}
  {GitHub Foundations}
  {GitHub}
  {Validação oficial de competência em GitHub, cobrindo práticas
  essenciais de colaboração, gerenciamento de projetos, segurança e
  fluxo de trabalho Open Source.}
  {}

\vspace{10pt}

\cvevent
  {Fevereiro 2024}
  {Cybersecurity Essentials}
  {The Linux Foundation}
  {Certificação focada nos fundamentos de segurança cibernética
  abrangendo princípios de defesa, criptografia e segurança
  operacional.}
  {}

\vspace{10pt}

\cvevent
  {Jan 2023}
  {Shell Scripting: Automação de Infraestrutura}
  {Alura}
  {Desenvolvimento de scripts para automação de tarefas de sistema,
  manipulação de streams de dados e administração eficiente de ambientes
  Linux.}
  {}

\vspace{10pt}

\cvevent
  {Jan 2023}
  {Arquitetura de Computadores: Fundamentos de Hardware}
  {Alura}
  {Estudo aprofundado sobre como os programas interagem com o hardware,
  ciclo de instruções da CPU e hierarquia de memória.}
  {}

\vspace{10pt}

\cvevent
  {Out 2022}
  {Formação C: Programação de Baixo Nível}
  {Alura}
  {Manipulação direta de memória, uso de ponteiros e compreensão da base
  estrutural dos sistemas operacionais modernos.}
  {}

\vspace{10pt}

\cvevent
  {Ago 2022}
  {Formação Java com Orientação a Objetos}
  {Alura}
  {Consolidação dos pilares da Orientação a Objetos (Encapsulamento,
  Herança, Polimorfismo) e aplicação de boas práticas de design de
  código no ecossistema Java.}
  {}

\vspace{10pt}

\cvevent
  {Out 2022}
  {Linux Onboarding: Operação Eficiente via CLI}
  {Alura}
  {Domínio da interface de linha de comando (CLI) para navegação, gestão
  de processos e permissões em sistemas Unix-like.}
  {}

\vspace{10pt}

\cvevent
  {Set 2022}
  {Desenvolvimento Web com PHP}
  {Alura}
  {Manipulação avançada de estruturas de dados (Arrays/Strings) e
  implementação de lógica backend para aplicações web dinâmicas.}
  {}

\vspace{10pt}

\cvevent
  {Ago 2022}
  {Linguagem Java Avançado: POO}
  {Fundação Bradesco}
  {Especialização em Programação Orientada a Objetos, focando em
  encapsulamento, herança, polimorfismo e tratamento de exceções.}
  {}

\vspace{10pt}

\cvevent
  {Dez 2021}
  {Fundamentos de Marketing Digital e Negócios}
  {Google}
  {Visão estratégica sobre métricas de desempenho digital e presença
  online, complementando o perfil técnico com visão de produto.}
  {}

\end{rightcolumn}

%------------------------------------------------------ Right Column ---
\end{paracol}
\end{document}
